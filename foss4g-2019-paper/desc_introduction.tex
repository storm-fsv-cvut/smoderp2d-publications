Erosion / hydrological models (EH) are being used for various research or engineering purposes. 
Results of such a model may be used as an input information for a planning 
or designing soil conservation measures. 
Volumes of runoff water and transported soil amounts
or time series of runoff water are being calculated
in order to design the protection measures sufficient for a given flood 
or soil transport event. Another example of a practical 
application of EH models may be land-use change or build up areas development studies 
and effect of those on water or soil transport. 
Great use of EH models is also in extreme event 
forecasting. For research purposes EH models 
are being used to proof a new theory or to test a hypothesis of mechanisms controlling the runoff and erosion.

Empirical erosion models are often based on Universal Soil Loss Equation (USLE), 
\cite{wischmeier1978,renard1997} and empirical hydrological models on Curve 
number method (CN) \cite{cronshey1986}, concepts more than 30 years old. 
Using empirical approaches may introduce limitations in designing the protection measures e.g. 
because those models does not take into account transient nature of those processes. 
Physically based models are being developed to overcome the empirical models limitations. 

Surface processes are spatially distributed, which is the reason why GIS 
is often deploying in the modeling process taking advantage 
of ready to use GIS features. EH models have similar structure 
(although each model is specific in terms in processes solved, 
its purpose or coding strategy). The forcing of the runoff and soil loss is precipitation
which is often introduced in the model in spatially distributed 
manner. Majority of models include an infiltration routine with 
spatially distributed parameters, since grass land 
and parking lot may be presented in single hydrological model 
and have vastly distinct infiltration characteristic. Infiltrated 
water is transported to the soil with different transport properties. 
Ponging water creates overland flow which leads to a soil transport 
and may case sever soil and nutrition losses in the landscape. 
In the modeled system may be presented linear (water courses, streets, ditches) 
or points (typically a water pump) features which affect the water 
flow and soil transport routines. GIS software has tools 
to operate with the linear or point features. 


The EH model may encounter with some run-time issues which 
rise from model spatial and temporal discretization. 
Data availability and larger computation resources lead more often 
to the use of finer spatial resolution. 
It was noted in \cite{molnar2000} that raster grid cell size is interchangeable 
in terms of spatial discretization if the model parameters 
were calibrated on the model with the same raster grid size. 
Finer spatial resolution, in some 
cases, causes problems with time discretization. Time step size is commonly 
controlled with Courant–Friedrichs–Lewy (CFL) criterion. 
CFL criterion force the time step to decrease if: a) velocity of flow 
process increases or; b) the spatial discretization become finer. 
Maximum acceptable CFL value, which preserve computation stability, theoretically equals one. 
For shallow surface processes (processes which take place in the used model)
CFL criterion should 
be even smaller than one as it was noted in \cite{zhang1989}
or \cite{esteves2000}. The need for smaller time step, to satisfy CFL criterion, 
is caused mainly by discrepancy between solution height 
(surface water) and cell size and surface roughness coefficient 
or sharp surface slope changes between adjacent cells. 

In case of EH models the commonly computed processes are sheet (water covers the earth's surface evenly)
and rill flow (water detaches the soil material and concentrates its flow in the created rill). 
Although the concentrated rill flow is particularly fast (causing  
the time step size constrain) it is usually not spatially extensive 
over the watershed. The rill flow usually occurs at concentrated 
flow paths. The computation may end up in a situation where a 
small portion of the computation domain force shorter time step 
(due to rill flow)  whereas the rest of the area allows a large one. 
In that case only a small part of the computed watershed causes the long model run-time. 


Nevýhoda -  Výpočet je numericky náročný, kuli CFL
Procesy jsou rychlé jen v některých buňkách, ostatní části “čekají”
Jedním z přístupů je využití rozbusnosti výpočetní techniky. 
TO je umožněno paralezlazcí výpočtu (CPU/GPU). Princip pravidelné mřížky -  nejsnazší, princip      subpovodí     

Testování jak využít paralelizaci je testováno na modelu SMODERP..
