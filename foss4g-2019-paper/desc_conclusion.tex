This manuscript presents SMODERP2D project and recently triggered 
development. Complete SMODERP2D source code is available on GitHub 
\cite{xxx} under GNU GPL licence. SMODERP2D computational tools have 
been successfully integrated into Esri ArcGIS, GRASS GIS and QGIS desktop GIS platforms. On top of that, the concept of so-called GIS providers allows publishing the SMODERP2D model as web processing service. Establishing experimental SMODERP2D 
OGC Web Processing Service is planned for 2019. Ongoing development 
is mainly focused on computational routines and parallel computations experiments. All the tools are currently distributed as experimental and are available for testing and user feedback.
The~official stable release of SMODERP2D model is planned in 2020. This 
includes also user documentation which is currently under development. 
In the 
case of SMODERP2D model, the run-time is an issue, especially 
if multiple mid-scale hydrological basins in fine spatial 
resolution grid computation needs to be undertaken. The code parallelization
is a common practice in cases where 
the reduction of run-time is convenient or even necessary, therefore
the existence of the TensorFlow-based branch; and although this branch is still
under development, the reduction of the computation costs is already reaching up to 40 per
cent depending on the data and architecture. This experiment also shows that the parallelized branch should not be used as the default one, but an ad hoc solution should be chosen depending on the data and available computing power. Even 
though the SMODERP2D model does not belong in the family of forecasting 
models (where the short run-time is necessary)  the run-time speed 
up will increase the usability of the model in practice and research applications. 
