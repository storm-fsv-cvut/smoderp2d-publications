\subsection{SMODERP2D model}
The SMODERP2D model has a simple structure based on mass balance equation:
\begin{equation}\label{equ:mass_bal}
    \frac{Storage}{\Delta t} = \nonumber  
    Inflow - Outflow
\end{equation}
In the equation~(\ref{equ:mass_bal}) the Storage is represented by
surface water level which $h$ [$L$] which changes each proceeding time
during the computation.

The $Inflow$ and $Outflow$ term on the right hand side of
equation~(\ref{equ:mass_bal}) consist of several components and it is
defines for each $i_th$ cell in the computation domain:
\begin{equation}\label{equ:inflow}
    Inflow_i = es_{i} + \sum_j^n q_{j}
\end{equation}
\begin{equation}\label{equ:outflow}
    Outflow_i = inf_{i} - q_{i} - ret_i
\end{equation}
\begin{tabbing} 
where \hspace{0.6cm} \= $es$ = effective precipitation\\
\> $q$ = inflow to resp. outflow from a given raster cell\\
\> $inf$ = infiltration\\
\> $ret$ = surface retention for a given raster cell
\end{tabbing}
The sum $\sum_j^n$ indicates that there can be more that one cell
flowing in cell i. The flow direction and therefore the sum $\sum_j^n$
%% ML: pridat referenci, kde byl D8 poprve publikovan?
is controlled be $D8$ flow direction algorithm.  Effective
precipitation $es$ is a potential precipitation reduced by
interception of the rainfall water on the vegetation.

%% ML: pridat referenci (?)
The model is forced to satisfy the Courant–Friedrichs–Lewiho (CFL)
criteria:
\begin{equation}\label{equ:CFL}
    CFL = \frac{q\textrm{d}t}{\textrm{d}x} < 1.0
\end{equation}
\begin{tabbing} 
where \hspace{0.6cm} \= $\textrm{d}t$ = time step\\
\> $\textrm{d}x$ = grid cell size
\end{tabbing}
If the flow $q$ in the cell highs the model is forced to decrease the
time step in order to satisfy the CFL criteria, since grid cell size
is fixed.

The flow $q$ in equation~(\ref{equ:inflow}) and~(\ref{equ:outflow})
has two components. Slower and spatially extensive sheet flow
$q_{sh}$:
\begin{equation}\label{equ:sheetflow}
    q_{sh} = XI^Yh^b
\end{equation}
\begin{tabbing} 
where \hspace{0.6cm} \= $X,Y,b$ = empirical parameters\\
\> $I$ = surface slope
\end{tabbing}
and faster concentrated rill flow $q_{rl}$ calculated by Mannings formula:
\begin{equation}\label{equ:rillflow}
    q_{rl} = A\frac{1}{n} R^{2/3} I^{1/2}
\end{equation}
\begin{tabbing} 
where \hspace{0.6cm} \= $A$ = cross-section area\\
\> $n$ = roughness in the rill\\
\> $R$ = hydraulic radii
\end{tabbing}
Resulting flow is a sum of sheet and rill flow:
\begin{equation}\label{equ:flow}
    q = q_{rl} + q_{rl}
\end{equation}

%% ML: pridat referenci (?)
Infiltration is solved with Phillip's infiltration equation:
\begin{equation}\label{eq:Phillips}
    inf = 1/2St^{-1/2} + Ks
\end{equation}
\begin{tabbing} 
where \hspace{0.6cm} \= $S$ = sorptivity\\
\> $K_s$ = hydraulic conductivity
\end{tabbing}

The reason why the parallelization is considered is the flow
in~(\ref{equ:flow}). Parameters of
relations~(\ref{equ:sheetflow})~(\ref{equ:rillflow}) and
(\ref{eq:Phillips}) may be spatially distributed. And the data are
handled by Esri ArcGIS, QGIS and GRASS GIS providers, see
\ref{ref:open_source_providers}.
