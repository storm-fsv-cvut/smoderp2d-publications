\subsection{Parallel computing experiments}

Because one of the most crucial points of SMODERP2D computations is
the speed, an experimental branch allowing (both CPU and GPU-based)
parallelised computations has been developed.

The main step was to rewrite all loop-based computations into
matrix-based mathematical operations. To keep matrices as so-called
tensors and to perform all the operations, an open source python
%% TODO: Ondra
library TensorFlow developed by Google Brain Team was used \cite{xxx}.

Even though TensorFlow is most widely used for machine learning and
its performance on basic mathematical operations is not always better
than the one of NumPy, it had been preferred for its easy switch
between CPU and GPU-based core (it depends only on the version of
TensorFlow the user has installed, no needs for changes in code) and
therefore support also for users without an access to machines with
GPU. Another advantage of TensorFlow is its usage of so-called
graphs. A graph is a representation of all operations in
dataflow/workflow and its individual operations are automatically sent
to multiple cores in a CPU or multiple threads in a GPU. These nodes
are run independently in parallel.

To support further development of TensorFlow and exploit its bleeding
edge functionalities, TensorFlow 2.0, which is published currently
just as an alpha version, was used in the SMODERP2D experimental
branch. Because TensorFlow 2.0 is still not suitable with all the
Python acrobatic tricks, NumPy was used for matrix operations in
places where TensorFlow could not (on places where loops were still
needed; looping through a NumPy array is incomparably faster than
through a Tensor).

This experimental SMODERP2D branch is still under development;
however, the alpha-version is ready to be used. The following table
presents the results of different tests made on this version:

%% TODO: Ondra
TODO

\begin{table}[h]
  \centering
  \begin{tabular}{|l|c|c|}\hline
    Setting&\multicolumn{2}{c|}{A4 size paper}\\\hline
    &mm&inches\\
    Top&25&1.0\\
    Bottom&25&1.0\\
    Left&20&0.8\\
    Right&20&0.8\\
    Column Width&82&3.2\\
    Column Spacing&6&0.25\\\hline
  \end{tabular}
  \caption{Margin settings for A4 size paper}
  \label{tab:Margin_settings}
\end{table}

\subsubsection{Further ideas for a basins-based parallel computing}

TODO
