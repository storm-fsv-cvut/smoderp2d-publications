

%----------------------------------------------------------------------------------------
%   Summary
%----------------------------------------------------------------------------------------



\begin{alertblock}{Summary}
  \begin{wrapfigure}[8]{r}{0.4\textwidth}
    \begin{minipage}{\dimexpr\linewidth-2\fboxrule-2\fboxsep}
    {\tiny
    \begin{lstlisting}
       @ @ @   @       @     @ @     @ @ @     @ @ @ @  @ @ @    @ @ @
      @        @ @   @ @   @     @   @     @   @        @     @  @     @
      @        @   @   @  @       @  @      @  @        @     @  @     @
        @ @    @       @  @       @  @      @  @ @ @    @ @ @    @ @ @
            @  @       @  @       @  @      @  @        @   @    @
            @  @       @   @     @   @     @   @        @    @   @
       @ @ @   @       @     @ @     @ @ @     @ @ @ @  @     @  @

      \  \  /   / /    \   \  /   \  /    /     /        @ @ @   @ @ @
       \ _\/   /_/      \   \/     \/    /_____/        @     @  @     @
           \__/          \  /      _\___/                     @  @      @
               \____      \/      /                          @   @      @
                    \_____/______/                         @     @      @
                                 \                       @       @     @
                                  \____________________ @ @ @ @  @ @ @
    \end{lstlisting}
    }
    \end{minipage}
  \end{wrapfigure}
  In the past parameters of the phy\-sically-based, distributed, event runoff/erosion SMODERP2D Model were inferred for each soil textural class. This approach were performed to simplify practitioners life since the parameters were ready to use if one have the soil texture of the area of interest. In order to assess variability of each parameter within and across textural classes the model were optimized to laboratory artificial rainfall experiments at 4 different textural classes. 
  Results shows that ...
\end{alertblock}\vspace{0.9cm}



%----------------------------------------------------------------------------------------
%   INTRODUCTION
%----------------------------------------------------------------------------------------


\mojesekce{Introduction and Methods}
\subsection{gov}
\begin{block}{Model structure}
    Based on a balance equation 
    $$
        \frac{Storage}{\Delta t} = \nonumber  
        Inflow - Outflow
    $$
    is inferred bucket model with kinematic wave approach for momentum 
    $$
%       \begin{split}
        \frac{\partial h_{i}}{\partial t} =  es_{i} + \sum_j^n q_{j} - inf_{i} - q_{i} - ret_i
    $$
    where $h$ is water level [$L$], $es$ is effective precipitation [$L.T^{-1}$], $inf$ is infiltration [$L.T^{-1}$], $ret$ surface retention [$L$], and $q$ is the Manning-Strickler formula
    \begin{equation}
      q = Xi^Yh^b. 
      \label{eq:manning}
    \end{equation}

    Parameters $X$, $Y$, and $b$ were inferred for each soil textural class. Variability of those parameters within and among textural classes is assessed in this poster. 
\end{block}





\subsection{Soil classes and ARE overview}
\begin{block}{Soil classes and ARE overview}
    \begin{itemize}
        \item 266 experimets with artificial rainfall were performed 
        \item Various slopes and rainfall intensities were tested for each location.
    \end{itemize}

    \begin{columns}
        \begin{column}{0.4\textwidth}
            \includegraphics[width=1.05\textwidth]{obr/soil_triangle.png}
        \end{column}
        \begin{column}{0.6\textwidth}
            {\small 
            \begin{table}[]
            \caption{ Overview of artificial rainfall experimets}
                \begin{tabular}{lcccccc}
                \hline
                \hline
                location      & year    & no. of            & \multicolumn{3}{c}{soil texture {[}\%{]}}  & soil class      \\
                            &         & exper.       & clay  & silt  & sand &  \\
                \hline
                Horoměřice    & 2002    & 25                & 25             & 58            & 17            & silty loam      \\
                Třebsín I     & 2004    & 22                & 5              & 60            & 35            & silty loam      \\
                Neustupov     & 2006    & 14                & 4              & 41            & 55            & sandy loam      \\
                Klapý         & 2007    & 25                & 30             & 54            & 16            & silty clay loam \\
                Třebsín II    & 2008    & 28                & 5              & 60            & 35            & silty loam      \\
                Třebešice I   & 2009    & 27                & 4              & 25            & 71            & sandy loam      \\
                Třebešice II  & 2010    & 36                & 7              & 46            & 47            & sandy loam      \\
                Nučice        & 2011    & 35                & 14             & 57            & 29            & silty loam      \\
                Všetaty I     & 2012    & 24                & 22             & 42            & 36            & loam            \\
                Všetaty II    & 2013    & 17                & 22             & 42            & 36            & loam            \\
                Třebešice III & 2014    & 22                & 8              & 56            & 36            & silty loam      \\
                Nové Strašecí & 2015    & 20                & 27             & 41            & 32            & loam            \\
                Řisuty        & 2017    & 21                & 18             & 34            & 48            & loam           \\
                \hline
                \hline
                \end{tabular}
            \end{table}
            }
        \end{column}
    \end{columns}
\end{block}


\subsection{Modeling exercises}
\begin{block}{Modeling exercises}
    \begin{columns}[t]
        \begin{column}{0.3\textwidth}
            {\bf Model optimization}
            \begin{itemize}
                \item infiltration parameters fitted with data
                \item differential evolution were used
                \item eq. \ref{eq:manning} parameters were optimized
                \item each experiment was optimized separately
            \end{itemize}
        \end{column}
        \begin{column}{0.3\textwidth}
            {\bf Sensitivity analyses:}
            \begin{itemize}
                \item screening: each parameter changed with factor and compared to optimized set of parameters 
                \item sum of squares for comparison
                \item sensitivity of eq. \ref{eq:manning} and infiltration parameters were tested
            \end{itemize}
        \end{column}
        \begin{column}{0.3\textwidth}
            {\bf Uncertainty analyses}
            \begin{itemize}
                \item 10000 Monte Carlo simulations for each experiment
                \item random sampling in the parameter space
                \item Nash Sutcliffe model efficiency for model run comparison
            \end{itemize}
        \end{column}
    \end{columns}
    Functionalities of python package Scipy were used to perform optimization \cite{scipy}. GNU Parallel software was used during optimization and Monte Carlo simulations \cite{Tange2011a}. 
\end{block}









        
