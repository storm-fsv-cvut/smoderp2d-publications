Erosion / hydrological models (EH) are being used for various research or engineering purposes. 
Results of such models may be used as input information for planning 
or designing soil conservation measures in the landscape and hydrological units - basins. 
Runoff water volume and transported soil amounts
or discharge time series are being calculated
in order to design the protection measures sufficient for a given flood 
or soil transport event. Another example of a practical 
application of EH models may be land-use change, build up areas development studies 
or effect of those on water or soil transport regime. 
Great use of EH models is also in extreme event 
forecasting. In research, EH models 
are being used to proof a new theory or to test hypotheses related 
to mechanism controlling the runoff and soil transport.

\vskip 1em

Empirical erosion models are often based on Universal Soil Loss Equation (USLE), 
\cite{wischmeier1978,renard1997} and empirical hydrological models on the Curve 
number method (CN) \cite{cronshey1986}, concepts more than 30 years old. 
Using empirical approaches may introduce limitations in the protection measures design e.g. 
because mentioned models do not take into account the transient nature of modelled processes. 
Physically based models are being developed to overcome the empirical models limitations. 

\vskip 1em

Processes taking place in a landscape are spatially distributed, which is the reason why GIS (Geographic Information System)
is often deploying in the modelling process taking advantage 
of ready to use GIS features. EH models have similar structure 
(although each model is specific in the terms of processes solved,
its purpose or coding strategy). Runoff and soil loss are initiated by precipitation
which is, especially for larger areas, spatially distributed process. 
Majority of models include an infiltration routine with 
spatially distributed parameters, since grassland 
and parking lot may be presented in a single hydrological model 
and have vastly distinct infiltration characteristic. Infiltrated 
water is transported to the soil which has varying transport properties. 
Ponging water creates overland flow which leads to a~soil transport 
and may cause severe soil and nutrition losses in the landscape. 
Linear (water courses, streets, ditches) 
or points (typically a water pump) features may be presented  
in the modelled system and affect the water flow or soil transport regime. 
GIS software has tools to operate with the linear and point features, 
and geospatial data which simplifies modellers live.

The EH model may encounter with some run-time issues which 
rise from model spatial and temporal discretization. 
Data availability and larger computation resources lead more often 
to the use of finer spatial resolution. 
It was noted in \cite{molnar2000} that raster grid cell size is interchangeable 
in the terms of a spatial discretization if the model parameters 
were calibrated on the model with the same raster grid size. 
Finer spatial resolution, in some 
cases, causes problems with a time discretization and the time step size. 
Time step size is commonly 
controlled with Courant--Friedrichs--Lewy (CFL) criterion~\cite{courant1928}. 
CFL criterion forces the time step to decrease if: a) velocity of flow 
process increases or; b) the spatial discretization becomes finer. 
Maximum acceptable CFL value, which preserves computation stability, theoretically equals one. 
For shallow surface processes (processes which take place in the used model)
CFL criterion should 
be even smaller than one as it was noted in \cite{zhang1989}
or \cite{esteves2000}. The need for smaller than one CFL criterion 
is caused mainly by the discrepancy between a solution (surface water height), 
cell size and surface roughness coefficient 
or by sharp surface slope changes between adjacent cells. 

In the case of EH models, the commonly computed processes are sheet 
and 
rill flow. 
The sheet flow covers the earth's surface evenly, whereas rill flow detaches 
the soil material and concentrates its flow in the created rill 
(therefore it is also called concentrated flow).
Although the concentrated rill flow is particularly fast (causing  
the time step size constraint) it usually occupies a small portion of the area. 
The computation may end up in a situation where a 
small portion of the computed area demands a shorter time step 
(due to rill flow presence) whereas the rest of the area allows larger time step. 
In that case, only a small part of the computed area with developed rills causes a~long model run-time. 

To summarize and outline the objectives of this manuscript. The advantage of high-resolution
geospatial data availability is constrained with an increasing computation demands
of a calculation. In the case of this manuscript, the extremely 
short time steps caused by the needs of CFL criterion is the main concern. 
Not all computed processes need a shorter time step and processes which are spatially limited (the concentrated flow in rill). 
In other words, 
the whole basin computation run-time is being increased due to a small part
of the computed basin. One way to overcome this problem is to use 
GPU or CPU-based parallelization. In this manuscript, TensorFlow Python library 
\cite{tensorflow2015-whitepaper} was tested to parallelize the EH model. 
Besides the TensorFlow also a CPU-based parallelization is outlined. 
The testing was performed with the SMODERP2D EH model. The model calculates the surface runoff
and soil loss processes with the use of GIS software for the data pre- and postprocessing. 
GRASS GIS provider and QGIS plugin were lately implemented in the SMODERP2D project, next to the already existing Esri ArcGIS Toolbox. 
Those new features and some of the principles used in the SMODERP2D model are also presented in this manuscript. 

% Nevýhoda -  Výpočet je numericky náročný, kuli CFL
% Procesy jsou rychlé jen v některých buňkách, ostatní části “čekají”
% Jedním z přístupů je využití rozbusnosti výpočetní techniky. 
% TO je umožněno paralezlazcí výpočtu (CPU/GPU). Princip pravidelné mřížky -  nejsnazší, princip      subpovodí     
% Testování jak využít paralelizaci je testováno na modelu SMODERP..
