\subsection{SMODERP2D model}
The SMODERP2D model has been integrated in open source 
GIS packages and tested for the GPU/CPU parallelization
within presented work. The model, which is now capable of 2D
calculation, has been developed from the 1D profile version~\cite{Holy1984}. 
Description of the model follows.

The model has a simple structure based on the mass balance equation:
\begin{equation}\label{equ:mass_bal}
    \frac{Storage}{\Delta t} = \nonumber  
    Inflow - Outflow
\end{equation}
where {\it Storage} represents 
surface water level $h$ [$L$] which changes each proceeding time
during the computation. $Inflow$ and $Outflow$ terms on the right-hand side of
the equation~(\ref{equ:mass_bal}) represent the water flowing in and out the storage 
during the time step ${\Delta t}$ and consist of several components.
The $Inflow$ and $Outflow$ of $i_{th}$ raster cell are defined as:
\begin{equation}\label{equ:inflow}
    Inflow_i = es_{i} + \sum_j^n q_{j}
\end{equation}
\begin{equation}\label{equ:outflow}
    Outflow_i = inf_{i} - q_{i} - ret_i
\end{equation}
\begin{tabbing} 
where \hspace{0.1cm} \= $es$ = effective precipitation [$LT^{-1}$]\\
\> $q$ = inflow to resp. outflow from a given cell [$LT^{-1}$]\\
\> $inf$ = infiltration [$LT^{-1}$]\\
\> $ret$ = surface retention for a given raster cell [$LT^{-1}$]
\end{tabbing}
The sum $\sum_j^n$ in the expression~(\ref{equ:inflow}) represents sum of all inflows to the cell $i$. 
The flow direction and therefore the sum $\sum_j^n$
is controlled by D8 flow direction algorithm~\cite{ocallaghan1984}.  Effective
precipitation $es$ is potential precipitation reduced by
interception of the rainfall water on the vegetation.

The model is forced to satisfy the Courant--Friedrichs--Lewy (CFL)
criterion~\cite{courant1928}:
\begin{equation}\label{equ:CFL}
    CFL = \frac{q\textrm{d}t}{\textrm{d}x} < 1.0
\end{equation}
\begin{tabbing} 
where \hspace{0.6cm} \= $\textrm{d}t$ = time step [$T$]\\
\> $\textrm{d}x$ = grid cell size [$L$]
\end{tabbing}
If the flow $q$ is high, the model is forced to decrease the
time step in order to satisfy the CFL criterion, since a grid cell size
is fixed.

The flow $q$ in the equations~(\ref{equ:inflow}) and~(\ref{equ:outflow})
has two components. Slower and spatially extensive sheet flow
$q_{sh}$:
\begin{equation}\label{equ:sheetflow}
    q_{sh} = XI^Yh^b
\end{equation}
\begin{tabbing} 
where \hspace{0.6cm} \= $X,Y,b$ = empirical parameters [$-$]\\
\> $I$ = surface slope [$-$]
\end{tabbing}
and faster concentrated rill flow $q_{rl}$ calculated by the Mannings formula:
\begin{equation}\label{equ:rillflow}
    q_{rl} = A\frac{1}{n} R^{2/3} I^{1/2}
\end{equation}
\begin{tabbing} 
where \hspace{0.6cm} \= $A$ = cross-section area [$L^2$]\\
\> $n$ = roughness in the rill [$TL^{-1/3}$]\\
\> $R$ = hydraulic radii [$L$]
\end{tabbing}
The resulting flow is a sum of sheet and rill flow:
\begin{equation}\label{equ:flow}
    q = q_{rl} + q_{rl}
\end{equation}
The sheet flow starts when the infiltration capacity is exceeded; 
when rainfall is higher than infiltration. The rill flow emerges if 
a critical water level of sheet flow is exceeded. The critical 
water level is defined based on critical shear stress; when the 
drag force of the flowing water becomes large than the cohesive 
forces of the soil particles.
From the definition, the sheet flow does not occur all over the 
basin area. The rill flow is usually presented to even lower extend. 
However, the CFL criterion is more likely constrained by the 
rapid rill flow even though its it occupy smaller area compared to sheet flow. 

Infiltration is solved with Phillip's infiltration equation \cite{philip1957theory}:
\begin{equation}\label{eq:Phillips}
    inf = 1/2St^{-1/2} + Ks
\end{equation}
\begin{tabbing} 
where \hspace{0.6cm} \= $S$ = sorptivity [$LT^{1/2}$]\\
\> $K_s$ = saturated hydraulic conductivity [$LT^{-1}$]
\end{tabbing}

Parameters of relations~(\ref{equ:sheetflow})~(\ref{equ:rillflow}) and
(\ref{eq:Phillips}) are in the most cases spatially distributed. It is 
therefore beneficial to incorporate GIS packages in the modeling process. 



